%+------------------------------------------------------------------------+
%| Frame: Lie infinity algebra Definition
%| Author: Antonio miti
%+------------------------------------------------------------------------+

\documentclass{standalone}
\usepackage{tikz}
\usepackage{verbatim}
\usetikzlibrary{arrows,shapes}
\usepackage{amsfonts,amsmath}
\usepackage{xcolor}


\begin{document}
	\tikzstyle{every picture}+=[remember picture]
	\everymath{\displaystyle}
	\tikzstyle{na} = [baseline=-.5ex]
	\begin{minipage}[c]{\linewidth}
		\begin{minipage}[t]{0.3\linewidth}
			\vspace{-.5em}
			\begin{displaymath}
								 \Big(
								 \tikz[baseline]{
								            \node[fill=blue!20,anchor=base] (t1)
								            {$ L$};
								        } 
									,
									 \tikz[baseline]{
								            \node[fill=blue!20,anchor=base] (t2)
								            {$ \lbrace \mu_k \rbrace_{k \in \mathbb{N}} $};
									}
								\Big)
				\end{displaymath}
				%	
		\end{minipage}
		%
		\begin{minipage}[t]{0.7\linewidth}
		   \tikz[na] \node[scale=0.5,coordinate,fill=blue!20,draw,circle] (n1) {};		    
				 $\mathbb{Z}$-Graded vector space 
				 \quad
				 $L = \bigoplus_{i\in\mathbb{Z}} L_i$	
			\\
			\tikz[na]\node [scale=0.5,coordinate,fill=blue!20,draw,circle] (n2) {};	    
				  	Family of homogenous skew-multilinear maps 
				  	\\(\emph{multi-brackets}) 
				  	\quad
				  	$\mu_k : \wedge^k L \rightarrow L[k-2]$
		\end{minipage}
		
		\begin{minipage}[t]{\linewidth}
		\footnotetext{
		satisfying \emph{"Higher Jacobi"} relations 
		~(	$\forall m\geq 1$ and $x_i$ homogeneous elements in $L$)
			\begin{displaymath}
			0 = \sum_{\substack{i+j=m+1\\ \sigma \in ush(i,m-i)}} 
			\kern-1em		
			{\color{gray}(-)^{i(j+1)}	(-)^\sigma \epsilon (\sigma; x) }
			~
			\mu_j\biggr( \mu_i \Big(x_{\sigma_1},\dots, x_{\sigma_i}\Big), x_{\sigma_{i+1}},\dots, x_{\sigma_m}\biggr)
			\end{displaymath}
		}	
		\end{minipage}
	\end{minipage}




% Now it's time to draw some edges between the global nodes. Note that we
% have to apply the 'overlay' style.
\begin{tikzpicture}[overlay]
        \path[->] (n1) edge [bend right] (t1);
        \path[->] (n2) edge [bend left] (t2);
       % \path[->] (n3) edge [out=0, in=-90] (t3);
\end{tikzpicture}

\end{document}
