%- HandOut Flag -----------------------------------------------------------------------------------------
\makeatletter
\@ifundefined{ifHandout}{%
  \expandafter\newif\csname ifHandout\endcsname
}{}
\makeatother

%- D0cum3nt ----------------------------------------------------------------------------------------------
\documentclass[beamer,10pt]{standalone}   
%\documentclass[beamer,10pt,handout]{standalone}  \Handouttrue  

\ifHandout
	\setbeameroption{show notes} %print notes   
\fi

	
%- Packages ----------------------------------------------------------------------------------------------
\usepackage{custom-style}
\usetikzlibrary{positioning}
\usepackage{multicol}


%--Beamer Style-----------------------------------------------------------------------------------------------
\usetheme{toninus}
\usepackage{animate}
\usetikzlibrary{positioning, arrows}
\usetikzlibrary{shapes}

\begin{document}
%-------------------------------------------------------------------------------------------------------------------------------------------------



%-------------------------------------------------------------------------------------------------------------------------------------------------
\begin{frame}[fragile]{Compatibility between gauge transformations and comoment maps}
	%
	Consider $(M,\omega)$ \alert{symplectic mfd.}
	%
	\begin{center}
			\includestandalone[width=.8\textwidth]{Pictures/Frame_BigDiagram_symplectic_V2}
	\end{center}
	%
	\vspace{-1em}
	\begin{minipage}[t][1.7cm][t]{\textwidth}
	\begin{itemize}
		\only<1-4>{
			\item<2-> Given a Symp. mfd. $(M,\omega)$ there is a naturally associated Poisson algebra ...
			\item<3-> .\alert<+>{... and also a Lie Algebroid}.
			\item<4-> A Lie algebroid is a "controlled" $\infty$-dimensional Lie algebra;
		}
		\only<5-6>{
			\item<5-> Prequantization Bundle $S^1\hookrightarrow P \to M$ with connection $\theta$,
			\item<5-> "infinitesimal quantomorphisms" $Q(P,\theta):=\lbrace Y \in \mathfrak{X}(P)~|~ \mathcal{L}_Y \theta =0 \}$.
		}
		\only<7-11>{
		\item<7-> Consider a deformed structure $\tilde{\omega}= \omega + d B$ with $B\in C^\infty(M)$;
		\item<9-> There is a natural isomorphism in the Lie Alg.oids category,
		\item<11-> Considering $\mathfrak{g}\circlearrowleft M$ preserving $\omega$ and $\tilde{\omega}$ ...
		}
		\item<11-> Neglect the prequantization...
		\vspace{-1em} 
			\begin{displaymath}
				\begin{tikzcd}
					\Psi~:&[-1em] C^{\infty}(M)_\omega \ar[r,"\Psi"]& \Gamma(TM\oplus \mathbb{R})_\omega
					\\[-2em]
					& f \ar[r,mapsto] & \binom{\mathscr{v}_f}{f}
				\end{tikzcd}
			\end{displaymath}
	\end{itemize}
	\end{minipage}
	\vfill
	\tcbset{colback=white,
	colbacktitle=white,
	colframe=red!70!black,
	boxrule=1pt,
	colupper=red!70!black,
	arc=15pt,
	}
	\begin{minipage}[t][1.7cm][t]{\textwidth}
	\only<6>{ 
		\begin{tcolorbox}[enhanced,frame hidden,borderline={0.5pt}{0pt}{blue}]
			\color{blue}{
			The left square and right triangles commutes!
			}
		\end{tcolorbox}
	}
	\only<11>{
		\begin{tcolorbox}[enhanced,frame hidden,borderline={0.5pt}{0pt}{blue}]
			\color{blue}{
			The left square and right triangle commute!
			}
		\end{tcolorbox}
	}
	\only<12>{
		\vspace{-.75em}
		\begin{tcolorbox}[enhanced,frame hidden,borderline={0.5pt}{0pt}{blue}]
			\color{blue}{
			The central pentagon commutes!
			}
		\end{tcolorbox}
	\vfill
		\vspace{-.75em}
		\begin{center}
		\tcbox[enhanced,frame hidden,borderline={0.5pt}{0pt}{red,dashed}]{	
			\alert{
			\faQuestionCircle \qquad
				{What happens in the higher (n-plectic) case?}
			\qquad \faQuestionCircle		
			}
		}
		\end{center}
	}
	\end{minipage}
	%
\end{frame}
\note[itemize]{
	\item The horizontal embedding is  $f \mapsto (v_f,f)$;
	\item Vertical maps are also known as \emph{Gauge transformations}
	\item upshot: 
	\begin{enumerate}
		\item 
	\end{enumerate}
}
%-------------------------------------------------------------------------------------------------------------------------------------------------


\begin{frame}[fragile]{Compatibility between gauge transformations and comoment maps}
	%
	Consider $(M,\omega)$ \alert{symplectic mfd.}
	{ and \alert{\underline{prequantizable}} ($S^1$-bundle $P$, connection $\theta$)}
	%
	\begin{center}
			\includestandalone[width=.8\textwidth]{Pictures/Frame_BigDiagram_symplectic_V3}
	\end{center}
	%
	\vspace{-2em}
	\begin{minipage}[t][1.7cm][t]{\textwidth}
	\begin{itemize}
		\only<1-4>{
			\item<2-> Given a Symp. mfd. $(M,\omega)$ there is a naturally associated Poisson algebra ...
			\item<3-> .\alert<+>{... and also a Lie Algebroid}.
			\item<4-> A Lie algebroid is a "controlled" $\infty$-dimensional Lie algebra;
		}
		\only<5-6>{
			\item<5-> Prequantization Bundle $S^1\hookrightarrow P \to M$ with connection $\theta$,
			\item<5-> "infinitesimal quantomorphisms" $Q(P,\theta):=\lbrace Y \in \mathfrak{X}(P)~|~ \mathcal{L}_Y \theta =0 \}$.
		}
		\only<7-11>{
		\item<7-> Consider a deformed structure $\tilde{\omega}= \omega + d B$ with $B\in C^\infty(M)$;
		\item<9-> There is a natural isomorphism in the Lie Alg.oids category,
		\item<11-> Considering $\mathfrak{g}\circlearrowleft M$ preserving $\omega$ and $\tilde{\omega}$ ...
		}
		\item<12-> Neglect the prequantization...
		\vspace{-1em} 
			\begin{displaymath}
				\begin{tikzcd}
					\Psi~:&[-1em] C^{\infty}(M)_\omega \ar[r,"\Psi"]& \Gamma(TM\oplus \mathbb{R})_\omega
					\\[-2em]
					& f \ar[r,mapsto] & \binom{\mathscr{v}_f}{f}
				\end{tikzcd}
			\end{displaymath}
	\end{itemize}
	\end{minipage}
	\vfill
	\tcbset{colback=white,
	colbacktitle=white,
	colframe=red!70!black,
	boxrule=1pt,
	colupper=red!70!black,
	arc=15pt,
	}
	\begin{minipage}[t][1.7cm][t]{\textwidth}
	\only<6>{ 
		\begin{tcolorbox}[enhanced,frame hidden,borderline={0.5pt}{0pt}{blue}]
			\color{blue}{
			The left square and right triangles commutes!
			}
		\end{tcolorbox}
	}
	\only<11>{
		\begin{tcolorbox}[enhanced,frame hidden,borderline={0.5pt}{0pt}{blue}]
			\color{blue}{
			The left square and right triangle commute!
			}
		\end{tcolorbox}
	}
	\only<12>{
		\vspace{-.75em}
		\begin{tcolorbox}[enhanced,frame hidden,borderline={0.5pt}{0pt}{blue}]
			\color{blue}{
			The central pentagon commutes!
			}
		\end{tcolorbox}
	\vfill
		\vspace{-.75em}
		\begin{center}
		\tcbox[enhanced,frame hidden,borderline={0.5pt}{0pt}{red,dashed}]{	
			\alert{
			\faQuestionCircle \qquad
				{What happens in the higher (n-plectic) case?}
			\qquad \faQuestionCircle		
			}
		}
		\end{center}
	}
	\end{minipage}
	%
\end{frame}
\note[itemize]{
	\item The horizontal embedding is  $f \mapsto (v_f,f)$;
	\item Vertical maps are also known as \emph{Gauge transformations}
	\item upshot: 
	\begin{enumerate}
		\item 
	\end{enumerate}
}




%-------------------------------------------------------------------------------------------------------------------------------------------------
\end{document}
