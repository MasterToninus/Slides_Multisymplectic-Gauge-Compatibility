%- HandOut Flag -----------------------------------------------------------------------------------------
\makeatletter
\@ifundefined{ifHandout}{%
  \expandafter\newif\csname ifHandout\endcsname
}{}
\makeatother

%- D0cum3nt ----------------------------------------------------------------------------------------------
\documentclass[beamer,10pt]{standalone}   
%\documentclass[beamer,10pt,handout]{standalone}  \Handouttrue  

\ifHandout
	\setbeameroption{show notes} %print notes   
\fi

	
%- Packages ----------------------------------------------------------------------------------------------
\usepackage{custom-style}
\usetikzlibrary{positioning}
\usepackage{multicol}


%--Beamer Style-----------------------------------------------------------------------------------------------
\usetheme{toninus}
\usepackage{animate}
\usetikzlibrary{positioning, arrows}
\usetikzlibrary{shapes}

\begin{document}
%-------------------------------------------------------------------------------------------------------------------------------------------------


%-------------------------------------------------------------------------------------------------------------------------------------------------
\subsection{Poisson algebra and Lie algebroids}
%-------------------------------------------------------------------------------------------------------------------------------------------------
%-------------------------------------------------------------------------------------------------------------------------------------------------
\begin{frame}[fragile]{Embedding of the observables algebra in the Lie algebroid}
	Given a \alert{symplectic mfd.} $(M,\omega)$ ...
	\vfill
	\begin{center}
		\includestandalone[width=.95\textwidth]{Pictures/Frame_Embedding_Diagram_symplectic}
	\end{center}
	\vfill
	\begin{minipage}[t][8.5em][t]{\textwidth}
		\begin{itemize}
			\only<1-3>{
			\item<1-> \alert<1>{... there is a naturally associated Poisson algebra ...}
			\item<2-> \alert<2>{... and also a (standard twisted) Lie Algebroid}.
			\item<3-> A Lie algebroid is a "controlled" $\infty$-dimensional Lie algebra given (in this case) by
			\begin{displaymath}
					\left[\binom{x_1}{f_1},\binom{x_2}{f_2}\right]
					~=~
					\binom{[x_1,x_2]}{x_1(f_2)-x_2(f_1)-\omega(x_1,x_2)}
				\end{displaymath}
			}
			\item[]<4->
				\quad\\
				\begin{thmblock}[There exists an embedding of Lie algebras.]
					\begin{displaymath}
						\begin{tikzcd}
							\Psi~:&[-1em] C^{\infty}(M)_\omega \ar[r,"\Psi"]& \Gamma(TM\oplus \mathbb{R})_\omega
							\\[-2em]
							& f \ar[r,mapsto] & \binom{\mathscr{v}_f}{f}
						\end{tikzcd}		
					\end{displaymath}
				\end{thmblock}
		\end{itemize}
	\end{minipage}
\end{frame}
\note{}
%-------------------------------------------------------------------------------------------------------------------------------------------------


%-------------------------------------------------------------------------------------------------------------------------------------------------
\subsection{Compatibility between gauge transformations}
%-------------------------------------------------------------------------------------------------------------------------------------------------
%-------------------------------------------------------------------------------------------------------------------------------------------------
\begin{frame}[fragile]{Compatibility between gauge transformations and comoment maps}
	%
	Consider $(M,\omega)$ \alert{symplectic mfd.}
	%
	\begin{center}
			\includestandalone[width=.8\textwidth]{Pictures/Frame_BigDiagram_symplectic}
	\end{center}
	%
	\vspace{-1em}
	\vfill
	\begin{minipage}[t][8em][t]{\textwidth}
		\begin{itemize}
			\only<2>{
				\item<2-> Consider a second gauge-related symplectic structure on $M$
					\begin{displaymath}
						\tilde{\omega} = \omega + d B \qquad \text{with} \quad B\in \Omega^1(M).
					\end{displaymath}
			}
			\only<3-4>{
				\item<3-> There is a natural isomorphism in the Lie Alg.oids category \emph{($B$-transformation)}
					\begin{displaymath}
						\binom{x}{f} \mapsto \binom{x}{f-\iota_x B} ~.
					\end{displaymath}
			}
			\only<6-7>{
				\item<6-> Consider an infinitesimal group action $\mathfrak{g}\circlearrowleft M$ which is Hamiltonian w.r.t. both $\omega$ and $\tilde{\omega}$.
				\item<7-> let be $f:\mathfrak{g} \to C^\infty(M)_\omega$ and $\tilde{f}:\mathfrak{g} \to C^\infty(M)_{\tilde{\omega}}$ two comoment map s.t.
				\begin{displaymath}
					\tilde{f}(\xi) = f(\xi) - \iota_{\underline{\xi}}B
				\end{displaymath}
				}
			\item[]						
		\end{itemize}
	\only<5>{
		\vspace{-.75em}
		\begin{center}
		\tcbox[enhanced,frame hidden,borderline={0.5pt}{0pt}{red,dashed}]{	
			\alert{
			\faQuestionCircle \qquad
				{How can we close the left-hand side?}
			\qquad \faQuestionCircle		
			}
		}
		\end{center}
	}
	\only<8->{
		\vspace{-.75em}
		\tcbset{colback=white,
		colbacktitle=white,
		colframe=red!70!black,
		boxrule=1pt,
		colupper=red!70!black,
		arc=15pt,
		}
		\begin{tcolorbox}[enhanced,frame hidden,borderline={0.5pt}{0pt}{blue}]
			\color{blue}{
			Lemma: The central pentagon commutes!
			}
		\end{tcolorbox}
	\vfill
	}
	\only<9->{
		\vspace{-.75em}
		\begin{center}
		\tcbox[enhanced,frame hidden,borderline={0.5pt}{0pt}{red,dashed}]{	
			\alert{
			\faQuestionCircle \qquad
				{What happens in the higher (n-plectic) case?}
			\qquad \faQuestionCircle		
			}
		}
		\end{center}
	}
	\end{minipage}	
\end{frame}
\note[itemize]{
	\item The horizontal embedding is  $f \mapsto (v_f,f)$;
	\item Vertical maps are also known as \emph{Gauge transformations}
	\item upshot: 
	\begin{enumerate}
		\item 
	\end{enumerate}
}
%-------------------------------------------------------------------------------------------------------------------------------------------------



%-------------------------------------------------------------------------------------------------------------------------------------------------
\subsection{Geometric interpretation in pre-quantization}
%-------------------------------------------------------------------------------------------------------------------------------------------------
%-------------------------------------------------------------------------------------------------------------------------------------------------
\begin{frame}[fragile]{Geometric interpretation of the diagram}
	%
	Consider $(M,\omega)$ \alert{symplectic} and \alert{\underline{prequantizable}}.
	%
	\begin{center}
			\includestandalone[width=.8\textwidth]{Pictures/Frame_BigDiagram_prequantum}
	\end{center}
	%
	\vspace{-2em}
	\begin{minipage}[t][1.7cm][t]{\textwidth}
	\begin{itemize}
		\only<1-3>{
			\item<1-> Fix a Prequantization Bundle $S^1\hookrightarrow P \to M$ with connection $\theta$,
			\item<2-> "infinitesimal quantomorphisms" $Q(P,\theta):=\lbrace Y \in \mathfrak{X}(P)~|~ \mathcal{L}_Y \theta =0 \}$.
		}
		\item<4-> Embdedding through Atiah algebroid.
	\end{itemize}
	\end{minipage}
	\vfill
	\tcbset{colback=white,
	colbacktitle=white,
	colframe=red!70!black,
	boxrule=1pt,
	colupper=red!70!black,
	arc=15pt,
	}
	\begin{minipage}[t][1.7cm][t]{\textwidth}
	\only<3>{ 
		\begin{tcolorbox}[enhanced,frame hidden,borderline={0.5pt}{0pt}{blue}]
			\color{blue}{
			Lemma: The left square commutes!
			}
		\end{tcolorbox}
	}
	\only<5>{
		\begin{tcolorbox}[enhanced,frame hidden,borderline={0.5pt}{0pt}{blue}]
			\color{blue}{
			Lemma: The left square and right triangle commute!
			}
		\end{tcolorbox}
	}
	\end{minipage}
	%
\end{frame}
\note[itemize]{
	\item questo fatto puramente algebrico ha un'interessante interpretazione nell'ambito della prequantizzaazione  Relevance to Prequantization
	\item The horizontal embedding is  $f \mapsto (v_f,f)$;
	\item Vertical maps are also known as \emph{Gauge transformations}
	\item upshot: 
	\begin{enumerate}
		\item 
	\end{enumerate}
}
%-------------------------------------------------------------------------------------------------------------------------------------------------


\end{document}


%-------------------------------------------------------------------------------------------------------------------------------------------------




\end{document}
