%- HandOut Flag -----------------------------------------------------------------------------------------
\makeatletter
\@ifundefined{ifHandout}{%
  \expandafter\newif\csname ifHandout\endcsname
}{}
\makeatother

%- D0cum3nt ----------------------------------------------------------------------------------------------
%\documentclass[beamer,10pt]{standalone}   
\documentclass[beamer,10pt,handout]{standalone}  \Handouttrue  

\ifHandout
	\setbeameroption{show notes} %print notes   
\fi

	
%- Packages ----------------------------------------------------------------------------------------------
\usepackage{custom-style}
\usetikzlibrary{positioning}
\usepackage{multicol}
\DeclareMathOperator{\varv}{\mathscr{v}}


%--Beamer Style-----------------------------------------------------------------------------------------------
\usetheme{toninus}
\usepackage{animate}
\usetikzlibrary{positioning, arrows}
\usetikzlibrary{shapes}

\begin{document}
%-------------------------------------------------------------------------------------------------------------------------------------------------

%---------------------------------------------------------------------------------------------------------------------------------------------------
\subsection{Lie $\infty$-algebra of Observables}
\begin{frame}[fragile,t]{Lie $\infty$-algebra of Observables (higher observables) }
	Let be $(M,\omega)$ a $n$-plectic manifold.
	\begin{defblock}[$L_\infty$-algebra of observables ~\emph{(Rogers)}]
		\hspace{.25em} Is a cochain-complex $(L,\{\cdot\}_1)$ \\
		\vspace{-2.5em}
		\begin{center}
		\ifHandout
			\includestandalone{Pictures/Figure_Observables}	
		\else
			\includestandalone{Pictures/Frame_Observables}
		\fi				
		\end{center}
		\onslide<2->{
			\hspace{.25em} with $n$ (skew-symmetric) multibrackets $(2 \leq k \leq n+1)$\\
			\vspace{-1.5em}
			\begin{center}
				\includestandalone{Pictures/Equation_Multibracket}	
			\end{center}
		}
		%
	\end{defblock}
  	\vfill
	\onslide<3->{
		\emph{Higher analogue} of the \emph{Poisson algebra structure} associated to a symplectic mfd.
	\vfill
	\begin{columns}
		\hfill
		\begin{column}{.11\linewidth}	
			If $n>1$:
			
		\end{column}	
		\begin{column}{.8\linewidth}
		\begin{itemize}
			\item[\xmark] \textcolor{red}{we lose} :\quad multiplication of observables, Jacobi equation;
			%\\ \hspace*{4.25em} full-fledged Jacobi equation;
			\item[\cmark] \textcolor{green}{we gain} :\quad brackets with arities different than two,\\
			\hspace*{4.25em}
			 Jacobi equation \emph{up to homotopies}.
		\end{itemize}		
		\end{column}		
	\end{columns}
	}
  \end{frame}
 \note[itemize]{
	\item if symplectic manifolds are the symmetric take on mechanics, Poisson algebras are the algebraic counterpart.
 	\item A Lie algebra is associated to an ordinary symplectic manifold (its Poisson algebra).
	%(Underlying this is a Lie algebra, whose Lie bracket is the Poisson bracket.)
	Similarly, one associates an Lie-$n$ algebra to any $n$-plectic manifold.
 	% https://ncatlab.org/nlab/show/n-plectic+geometry 	 
 	 %https://ncatlab.org/nlab/show/Poisson+bracket+Lie+n-algebra
	 \item Basically, the higher observables algebra is a chunk of the de Rham complex of $M$ with inverted grading( convention employed here) and an extra structure called "multibrackets".
 	\item ( In the 1-plectic case it reduces to the corresponding Poisson algebra of classical observables)
 	\item Rogers associated to any n-plectic mfd a $L\-\infty$ algebra, Zambon generalized it to the pre-n-plectic case.
 	\item Recognize in the definition of $\{\cdot,\ldots,\cdot\}_k$ the contraction with hamiltonian fields $v_\sigma$ w.r.t. $\sigma$.
  	\item Note $	\iota_{v_{\sigma_1}}\cdots\iota_{v_{\sigma_k}} = (-)^{(k-1)+(k-2)+\dots+1}\iota_{v_{\sigma_k}}\cdots\iota_{v_{\sigma_1}} = (-)^{\frac{k(k-1)}{2}}\iota_{v_{\sigma_k}}\cdots\iota_{v_{\sigma_1}}$ 
 	The definition usually find in literature of Rogers multibrackets involves the coefficient $ (-)^{\frac{k(k-1)}{2}} = -\varsigma(k-1) = (-)^{k+1} \varsigma(k)$.
  \item higher observables is Special instance of a more general object  called $L\-\infty$ Algebra...
 }
%------------------------------------------------------------------------------------------------


%-------------------------------------------------------------------------------------------------------------------------------------------------
\subsection{Homotopy comomentum maps}\label{frame:hcmm-main}
\begin{frame}[fragile]{Homotopy comomentum maps}
	Consider a Lie algebra action $v:\mathfrak{g} \to \mathfrak{X}(M)$  \underline{preserving the $n$-plectic form $\omega$}.
	\vfill
	\begin{defblock}[Homotopy comomentum map \emph{(Callies, Fregier, Rogers, Zambon)}]
		\ifHandout
			\includestandalone{Pictures/Figure_Lifting}
		\else
			\includestandalone{Pictures/Frame_Lifting}
		\fi					
	\end{defblock}
	\onslide<4->{
	\begin{lemblock}[HCMM unfolded  \cite{Callies2016}]
			%
			HCMM is a sequence of (graded-skew) multilinear maps:
			\begin{displaymath}
				(f)  = \big\lbrace f_k: \; \Lambda^k{\mathfrak g} \to L^{1-k} \subseteq \Omega^{n-k}(M) 
				~\big\vert~ 0\leq k \leq n+1  \big\rbrace
			\end{displaymath}
			\emph{fulfilling:}%\emph{such that:}
			\begin{itemize}
				\item<5-> $f_0 = 0 $, $f_{n+1} = 0$
				\item<6-> $d f_k (p) = f_{k-1} (
				\tikz[baseline,remember picture]{\node[rounded corners,
                        fill=green!5,draw=green!30,anchor=base]            
            			(target) {$\partial $ };
            	}				
				p)  - (-1)^{\frac{k(k+1)}{2}} \iota(v_p) \omega 
				\qquad\scriptstyle \forall p \in \Lambda^k(\mathfrak{g}),\; \forall k=1,\dots n+1$
			\end{itemize}
		\onslide<7->{
			\tikz[overlay,remember picture]
			{
				\node[rounded corners,
	                 draw=green!30,anchor=base]            
	            	 (base) at ($(current page.east)-(3,3)$) [rotate=-0,align=center] {\footnotesize{\hyperlink{frame:CE-complex}{\emph{Chevalley-Eilenberg boundary op.}}}};
			}	
		\begin{tikzpicture}[overlay,remember picture]
	    	\path[->] (base.west) edge[bend right,green](target.north east);
	    \end{tikzpicture}
	    }
	\end{lemblock}	
	}
	\vfill
\end{frame}
\note[itemize]{
	\item  An infinitesimal symmetry is a lie algebra morphism such that $\mathcal{L}_{v_x} \omega = 0 ~ \forall x \in \mathfrak{g}$.
	\\ (It is also call an infinitesimal multisymplectic action and $v_x$ is the infinitesimal generator of the action, corresponding to $x \in \mathfrak g$.) 
	\item Essentially, admitting a comoment maps mean that $v$ acts via Hamiltonian vector fields.
	\item In mechanical terms, a moment map is a tool associated with a Hamiltonian action of a Lie group on a symplectic manifold, used to construct conserved quantities for the action.(see \ref{frame:HCMMandConserved} in appendix.
}
%-------------------------------------------------------------------------------------------------------------------------------------------------

%-------------------------------------------------------------------------------------------------------------------------------------------------
\subsection{Vinogradov Algebroids}
%-------------------------------------------------------------------------------------------------------------------------------------------------
\begin{frame}{Vinogradov Algebroids}
	\begin{defblock}[Vinogradov algebroid (higher Courant)]
		\includestandalone[width=0.95\textwidth]{Pictures/Figure_vinogradov}	
	\end{defblock}

	\begin{itemize}
		\item $(n=1) ~ \Rightarrow$ standard twisted \emph{Lie algebroid};
		\item $(n=2) ~ \Rightarrow$ standard twisted \emph{Courant algebroid};
	\end{itemize}

\end{frame}
\note[itemize]{
\item Att, questa definizione sarebbe lo standard twisted. E' possibile trovare in letteratura definizioni piu' generali corrispondenti alla nozione di Courant algebroid in astratto.
}
%-------------------------------------------------------------------------------------------------------------------------------------------------

%-------------------------------------------------------------------------------------------------------------------------------------------------
\begin{frame}{Vinogradov $L_\infty$-algebra}
	Vin. alg.oids are $NQ$-manifolds ($L_\infty$-algebroids).
	$\quad\Rightarrow\quad$ 
	Associated $L_\infty$-algebra.

	\begin{defblock}[Vinogradov $L_\infty$-algebra \cite{Zambon2012}]
		\includestandalone[width=0.95\textwidth]{Pictures/Figure_vinogradov-Linfty}	
	\end{defblock}
	\vfill
	\only<6->{
		\tikz[overlay,remember picture]
		{
			\node[rounded corners,
                 fill=gray!1,draw=gray!30,anchor=base]            
            	 (base) at ($(current page.south)+(0,.5)$) [rotate=-0,text width=10cm,align=center] { \footnotesize{\color{gray}{
            	 $e_i = \pair{X_i}{\alpha_i} \in \mathfrak{X}(M)\oplus \Omega^{n-1}(M)$ 
            	 \quad~,\qquad
            	 $f_i \in \bigoplus_{k=0}^{n-2}\Omega^k(M)$.
            	 }}};
		}			
	}

\end{frame}
\note[itemize]{
	\item The actions of non vanishing multi-brackets (up to permutations of the entries) on arbitrary vectors
are given in the slide.
	\item $\mu_2 \left(e_1,e_2\right) 	= [e_1,e_2]_\omega 
			= \pair{[X_1,X_2]}{\dd \langle e_1, e_2\rangle_- 
			+ (\iota_{X_1}\dd\alpha_2 - \iota_{X_2}\dd\alpha_1 + \iota_{X_1}\iota_{X_2}\omega)}$
	\item $				\mu_2 \left(e_1,f_2\right) = -\mu_2(f_2,e_1) = 
				\frac{1}{2} \mathcal{L}_{X_1} f_2 = \langle e_1, \dd f_2 \rangle_-$
	\item $k$-ary bracket for $k \ge 3$ an \emph{odd} integer:	 
		\begin{equation}\label{eq:VinoMultibrakAllaZambon_1}
			\begin{split}
				\mu_k(\varv_0,\cdots,\varv_{k-1})
				=&
				\left(\sum_{i=0}^{k-1} {(-)^{i-1}\mu_k(f_i+\alpha_i,X_0,\dots,\widehat{X_i},\dots,X_{k-1})}\right)
				+\\
				&+(-)^{\frac{k+1}{2}} \cdot k \cdot B_{k-1} \cdot 
				\iota_{X_{k-1}}	\dots \iota_{X_{0}} \omega			
				~;
			\end{split}
		\end{equation}
		where 
		\begin{equation}\label{eq:VinoMultibrakAllaZambon_2}
			\begin{split}
			\mu_k&(f_0+\alpha_0,X_1,\dots,X_{n-1}) =
			\\
			&=
			~c_k
			\sum_{1\le i<j\le k-1}(-1)^{i+j+1}\iota_{X_{k-1}}\dots   
  			\widehat{\iota_{X_{j}}}\dots \widehat{\iota_{X_{i}}}\dots
				\iota_{X_{1}} ~ [f_0+\alpha_0,X_i,X_j]_3~.
			\end{split}
		\end{equation}			
In the above formula,		$[\cdot,\cdot,\cdot]_3 = -T_0$ denotes the ternary bracket %$\mu_3$ 
associated to the untwisted ($\omega=0$) Vinogradov Algebroid, and $c_k$ is a numerical constant
		\begin{equation}\label{eq:UglyCoefficient}
			c_k= (-)^{\frac{k+1}{2}}\frac{12~B_{k-1}}{(k-1)(k-2)}.
		\end{equation}
}
%-------------------------------------------------------------------------------------------------------------------------------------------------


%-------------------------------------------------------------------------------------------------------------------------------------------------
\subsection{Rogers embedding}
%-------------------------------------------------------------------------------------------------------------------------------------------------
\begin{frame}[fragile]{Embedding observables $L_\infty$-algebra into Vinogradov $L_\infty$-algebra}
	Consider now $\omega$ \alert{$n$-plectic}
	\vfill
	\begin{center}
		\includestandalone[width=.8\textwidth]{Pictures/Frame_Embedding_Diagram_k-plectic_V1}
	\end{center}
	\vfill
	\only<1-3>{
		.
	}


	\only<4->{
	\begin{thmblock}[Embedding of $L_\infty$-algebras  $\Psi:L_\infty(M,\omega)\hookrightarrow L_{\infty}(E^n,\omega)$\quad \cite{Miti2021}.]
	\begin{itemize}[leftmargin=0pt]
		\item[$\cdot$]<4-> 
			consider the graded vector subspace $\mathcal{A}$
			\begin{displaymath}
			\mathclap{
			{\mathcal{A}^k} =
			\begin{cases}
		\left.\left\lbrace
		\binom{X}{\alpha}\in \mathfrak{X}(M)\oplus \Omega^{n-1}(M)
		~ \right\vert ~
		\iota_X \omega = -d \alpha\right\rbrace
&\quad k=0,\\
				\Omega^{n-1+k}(M) &\quad -n+1 \leq k < 0.
			\end{cases}			
			}
			\end{displaymath}						
		\item[$\cdot$]<5-> 
			restrict the two $L_\infty$-structures to $\pi$ and $\mu$ on $\mathcal{A}$
		\item[$\cdot$]<6->

			$L_\infty(M,\omega) \cong 
				(\mathcal{A},\pi) \color{blue}\cong\color{black}
				(\mathcal{A},\mu) \hookrightarrow
				L_\infty(E^n,\omega)$
	\end{itemize}
	\end{thmblock}
	}
	\only<6->{
		\tikz[overlay,remember picture]
		{
			\node[rounded corners,
                 fill=orange!1,draw=orange!30,anchor=base]            
            	 (base) at ($(current page.east)-(2.25,4)$) [rotate=-0,text width=3cm,align=center] { \footnotesize{\color{red}{
            	 Complete proof\\
            	 \faWarning ~ up to $n\geq 4$! ~ \faWarning 
            	 }}};
		}			
	}

		

\end{frame}
\note{}
%-------------------------------------------------------------------------------------------------------------------------------------------------


\subsection{Compatibility with Gauge transformations}
%-------------------------------------------------------------------------------------------------------------------------------------------------
\begin{frame}{Compatibility with Gauge transformations}
	Consider now $\omega$ \alert{n-plectic} \quad and \alert{$\tilde{\omega}=\omega + d B$}:
	\vfill
	%
	\begin{center}
		\includestandalone[width=.8\textwidth]{Pictures/Frame_Gauge_Diagram_k-plectic}
	\end{center}	
	%
	\vfill
	%
	

	\begin{itemize}
	\only<1-3>{
		\item<2-> Vinogradov alg.oids w.r.t cohomologous twisting closed forms are isomorphic.
		\item<3-> Induced isomorphism at the level of $L_\infty$-algebras
	}
	\only<4->{
		\item<4-> Consider a Lie algebra action $\mathfrak{g}\to \mathfrak{X}(M)$ admitting HCMM w.r.t $\omega$ and $\tilde{\omega}$
	}
	\end{itemize}

	\vfill
	\tcbset{colback=white,
		colbacktitle=white,
		colframe=blue!70!black,
		boxrule=1pt,
		colupper=blue!70!black,
		arc=15pt,
		}
	\onslide<5->{
	\begin{tcolorbox}[sidebyside,righthand width=.75\linewidth]
		Thm: \cite{Miti2021}
		\tcblower
		\color{blue}
The central square commutes. 
			\\\emph{(On the nose, not "up to homotopies")}.
	\end{tcolorbox}	
		\tikz[overlay,remember picture]
		{
			\node[rounded corners,
                 fill=orange!1,draw=orange!30,anchor=base]            
            	 (base) at ($(current page.east)-(1.75,4)$) [rotate=-0,text width=3cm,align=center] { \footnotesize{\color{red}{
            	 Complete proof\\
            	 \faWarning ~ up to $n\geq 4$! ~ \faWarning 
            	 }}};
		}		
	
	
	}
	

\end{frame}
\note[itemize]{
	\item Our results can be seen as a tiny step toward  undestanding the analogue of prequantization in the setting of multisymplectic geometry (hence field theory).
}
%-------------------------------------------------------------------------------------------------------------------------------------------------









%-------------------------------------------------------------------------------------------------------------------------------------------------
\end{document}