%+----------------------------------------------------------------------------+
%| SLIDES: 
%| Chapter: Complementary material - details on eventual questions
%| Author: Antonio miti
%| Event: PHD preliminary Defence
%+----------------------------------------------------------------------------+

%- HandOut Flag -----------------------------------------------------------------------------------------
\newif\ifHandout

%- D0cum3nt ----------------------------------------------------------------------------------------------
\documentclass[beamer,10pt]{standalone}   
%\documentclass[beamer,10pt,handout]{standalone}  \Handouttrue  

%- HandOut Flag -----------------------------------------------------------------------------------------
\ifHandout
	\setbeameroption{show notes} %print notes   
\fi

	
%- Packages ----------------------------------------------------------------------------------------------
\usepackage{custom-style}

%--Beamer Style-----------------------------------------------------------------------------------------------
\usetheme{toninus}



\providecommand{\blank}{\text{\textvisiblespace}}


\newcommand{\subsectiontitle}{
  \begin{frame}
  \vfill
  \centering
  \begin{beamercolorbox}[sep=8pt,center,shadow=true,rounded=true]{title}
    \usebeamerfont{title}\insertsectionhead\par%
    \usebeamerfont{title}\insertsubsectionhead\par%
  \end{beamercolorbox}
  \vfill
  \end{frame}
}

\providecommand{\blank}{\text{\textvisiblespace}}




%---------------------------------------------------------------------------------------------------------------------------------------------------
%- D0cum3nt ----------------------------------------------------------------------------------------------------------------------------------
\begin{document}
%------------------------------------------------------------------------------------------------

%##################################################################################
\begin{frame}
	\begin{center}
	\Huge\emph{Supplementary Material}
	\end{center}
\end{frame}
\note[itemize]{
	\item
}
\addtocounter{framenumber}{-1}
%##################################################################################





%===================================================================================
\section{Background}
%===================================================================================



%-------------------------------------------------------------------------------------------------------------------------------------------------
\subsection{MultiSymplectic Geometry}
%-------------------------------------------------------------------------------------------------------------------------------------------------

%-------------------------------------------------------------------------------------------------------------------------------------------------
\begin{frame}[fragile]{Multisymplectic geometry in a nutshell}
	\begin{block}{Historical motivation}
		Mechanics: geometrical foundations of \textit{(first-order)} field theories.
	\end{block}
	\vfill	
	\begin{table}
		\only<2>{
		\begin{tabular}{|p{0.2\textwidth}|p{0.3\textwidth}|p{0.35\textwidth}|} 
            \hline
            \parbox[][20pt][c]{0.2\textwidth}{mechanics} & \multicolumn{2}{c|}{geometry} \\
            \hline
            \parbox[][20pt][c]{0.2\textwidth}{phase space} & symplectic manifold &  \\[.25em]
            \parbox[][20pt][c]{0.2\textwidth}{classical \\ observables} & Poisson algebra &  \\[.25em]
            \parbox[][20pt][c]{0.2\textwidth}{symmetries} &  group actions admitting comoment map &  
            \\
            \hline
  \multicolumn{1}{c}{}
            &  \multicolumn{1}{@{}c@{}}{$\underbrace{\hspace*{.3\textwidth}}_{\text{point-like particles systems}}$} 
            &  \multicolumn{1}{@{}c@{}}{}              \\
		\end{tabular}
		
		
		}
		\onslide<3->{
		\begin{tabular}{|p{0.2\textwidth}|p{0.3\textwidth}|p{0.35\textwidth}|} 
            \hline
            \parbox[][20pt][c]{0.2\textwidth}{mechanics} & \multicolumn{2}{c|}{geometry} \\
            \hline
            \parbox[][20pt][c]{0.2\textwidth}{phase space} & symplectic manifold & multisymplectic manifold \\[.25em]
            \parbox[][20pt][c]{0.2\textwidth}{classical \\ observables} & Poisson algebra & $L_\infty$-algebra \\[.25em]
            \parbox[][20pt][c]{0.2\textwidth}{symmetries} &  group actions admitting comoment map & group actions admitting 
			\tikz[baseline,remember picture]{\node[rounded corners,
                        fill=orange!10,draw=orange!30,anchor=base]            
            			(target) {homotopy comomentum map};
            }
            \\
            \hline
  \multicolumn{1}{c}{}
            &  \multicolumn{1}{@{}c@{}}{$\underbrace{\hspace*{.3\textwidth}}_{\text{point-like particles systems}}$} 
            &
            \multicolumn{1}{@{}c@{}}{$\underbrace{\hspace*{.3\textwidth}}_{\text{field-theoretic systems}}$} 
               \\
		\end{tabular}
		}
	\end{table}		
	\vfill
	\onslide<4->{
	
	\begin{block}{Scope of the thesis}
		\begin{itemize}
			\item[$\bullet$] Develop theory of 
				\tikz[baseline,remember picture]{\node[rounded corners,
                        fill=orange!10,draw=orange!30,anchor=base]            
            			(base) {homotopy comomentum maps};
            	}
            \item[$\bullet$] produce new meaningful examples.
		\end{itemize}
	\end{block}


                    \begin{tikzpicture}[overlay,remember picture]
                    \path[->]<4-> (base.north east) edge[bend right](target.south east);
                    \end{tikzpicture}
	}


\end{frame}
\note[itemize]{
	\item Historically, the interest in multisymplectic manifolds, has been motivated by the need for understanding the geometrical foundations of first-order classical field theories.
	The key point is that, just as one can associate a symplectic manifold to an ordinary classical mechanical system (e.g. a single
point-like particle constrained to some manifold), it is possible to associate a multisymplectic
manifold to any classical field system (e.g. a continuous medium like a filament or a fluid). See frame Extra-\ref{Frame:Ms-Field-Mechanics} 
	
	\item General ideas basic parallelisms with caveats
	\item caveat: points in multiphase spaces are not states
	\item the table hides the duality between geometric and algebraic approaches to the problem.
	\item 
}
%-------------------------------------------------------------------------------------------------------------------------------------------------


%-------------------------------------------------------------------------------------------------------------------------------------------------
\begin{frame}[fragile]{Multisymplectic manifolds} %Fragile -->workaround tikzcd
	\begin{defblock}[$n$-plectic manifold ~\emph{(Cantrijn, Ibort, De Le\'on)}]
	\includestandalone[width=0.95\textwidth]{Pictures/Figure_multisym}	
	\end{defblock}
	%
	\begin{defblock}[Non-degenerate $(n+1)$-form]
		\begin{columns}
			\begin{column}{.45\linewidth}
				\centering{
				The $\omega^\flat$ (flat) bundle map is injective.
				}
			\end{column}
			\begin{column}{.5\linewidth}
						\vspace{-.5em}
				\[
				\begin{tikzcd}[column sep= small,row sep=0ex,
				/tikz/column 1/.append style={anchor=base east}]
				    \omega^\flat \colon T M \ar[r]& \Lambda^n T^\ast M \\
  						 (x,u) \ar[r, mapsto]& (x,\iota_{u} \omega_x)						
				\end{tikzcd}	
				\]
			\end{column}
		\end{columns}
	\end{defblock}
	%
	\pause
	\begin{defblock}[Hamiltonian $(n-1)$-forms]
		\begin{displaymath}
			\Omega^{n-1}_{ham}(M,\omega) 	:=
			\biggr\{ \sigma \in  \Omega^{n-1}(M) \; \biggr\vert \; 
				\exists \mathscr{v}_\sigma \in \mathfrak{X}(M) ~:~ 
				\tikz[baseline,remember picture]{\node[rounded corners,
                        fill=orange!5,draw=orange!30,anchor=base]            
            			(target) {$d \sigma = -\iota_{\mathscr{v}_\sigma} \omega$ };
            	}				
				~\biggr\} 
			\end{displaymath}
	\end{defblock}
	%
	%
	\pause
		\tikz[overlay,remember picture]
		{
			\node[rounded corners,
                 fill=orange!5,draw=orange!30,anchor=base]            
            	 (base) at ($(current page.east)-(2.25,1.8)$) [rotate=-0,text width=4cm,align=center] {\footnotesize{\textcolor{red}{Hamilton-DeDonder-Weyl \\equation}}};
		}	
	\begin{tikzpicture}[overlay,remember picture]
    	\path[->] (base.west) edge[bend left,red](target.south west);
    \end{tikzpicture}	
	\pause
	\vfill
	%
	\begin{block}{Examples:}
		\vspace{-.5em}
		\setbeamercovered{transparent}
		\begin{itemize}[<+->]
			\item[$\bullet$] $n=1$ \qquad\qquad\qquad $\Rightarrow$\quad $\omega$ is a symplectic form
			\item[$\bullet$]  $n=(dim(M)-1)$ \quad$\Rightarrow$\quad $\omega$ is a volume form
			%Any oriented $(n+1)$-dimensional manifold is $n$-plectic w.r.t. the volume form.
			%\item[$\bullet$] Let $G$ a semisimple Lie group and $\langle\cdot,\cdot \rangle$  its killing form. Then $\langle [\cdot,\cdot],\cdot \rangle$ extends to a biinvariant multisymplectic form $\omega$.
			\item[$\bullet$] Let $Q$ a smooth manifold, the multicotangent bundle $\Lambda^n T^\ast Q$ is naturally $n$-plectic.%
			\quad
			\textit{(cfr, \href{https://arxiv.org/abs/physics/9801019}{GIMMSY} construction for classical field theories)}
		\end{itemize}
	\end{block}			 
	
%
\end{frame}
\note[itemize]{
	\item Multisymplectic ($n$-plectic) geometry is a generalization of symplectic geometry where a closed, non degenerate $n+1$-form $(n\geq 1)$  takes the place of the symplectic 2-form
	
	\item multisymplectic means \emph{going higher} in the degree of $\omega$
	
	\item non degeneracy means $\iota_v\omega = 0 \Leftrightarrow v=0$.
	
	\item examples 
		\begin{itemize}
			\item[$\bullet$] 1-plectic $=$ symplectic
			\item[$\bullet$] Any oriented $(n+1)$-dimensional manifold is $n$-plectic w.r.t. the volume form.
			\item[$\bullet$] The multicotangent bundle $\Lambda^n T^\ast Q$ is naturally $n$-plectic.
		\end{itemize}
	
	\item We recognize the special class of forms, called Hamiltonian, determining the Hamiltonian vector fields. 
	Not every $n-1$ form admits an Hamiltonian vector field.
	When it exists, non degeneracy guarantees unicity.
	
	\item Observe also that, by degree reason, when $n$ is equal to $1$ or $dim(M)+1$ an injective flat map $\flat$ is also bijective.
	
	\item It is important to stress that mechanical systems are not the only source instances of this class of of structures. 
				e.g. any semisimple Lie groups has associated a 2-plectic structure and any oriented $n+1$ dimensional manifold is naturally $n$-plectic.
				

}
%---------------------------------------------------------------------------------------------------------------------------------------------------



































\subsection{References}
%------------------------------------------------------------------------------------------------
% https://en.wikibooks.org/wiki/LaTeX/Bibliographies_with_biblatex_and_biber
\begin{frame}[t,allowframebreaks]{Extended Bibliography}
	\bibliographystyle{alpha}
	\bibliography{bibfile}
\end{frame}
%------------------------------------------------------------------------------------------------


%------------------------------------------------------------------------------------------------
\end{document}
