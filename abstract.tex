\documentclass[10pt,a4paper]{article}
\usepackage[utf8]{inputenc}
\usepackage[english]{babel}
\usepackage{amsmath}
\usepackage{amsfonts}
\usepackage{amssymb}
\usepackage[left=2cm,right=2cm,top=2cm,bottom=2cm]{geometry}
\usepackage{hyperref}

\title{
Gauge transformations of multisymplectic manifolds and $L_\infty$ observables
}
\author{\href{https://dmf.unicatt.it/miti/}{Antonio Michele Miti}}
\date{
	Max Planck Institute for Mathematics
	\\[.2em]
	\today
}



\begin{document}

\maketitle
\begin{abstract}
Multisymplectic manifolds are a straightforward generalization of symplectic manifolds where closed non-degenerate k-forms are considered in place of 2-forms.
A natural theme that arises when dealing with both symplectic and multisymplectic structures is investigating the relationship between gauge-related multisymplectic manifolds, i.e. endowed with different multisymplectic forms lying in the same cohomology class.

This talk will focus on the L-infinity algebras of observables associated with a pair of gauge-related multisymplectic manifolds.
To date, no canonical correspondence is known between two gauge-related observables algebras.
However, it can be possible to exhibit a compatibility relation between those observables that are momenta of corresponding homotopy moment maps (the higher analogue of a moment map in the multisymplectic setting).

Although this construction is essentially algebraic in nature, it also admits a geometric interpretation when declined to the particular case of pre-quantizable symplectic forms. The latter case provides some evidence that this construction may be related to the higher analogue of geometric quantization for integral multisymplectic forms.

This talk is based on ongoing joint work with Marco Zambon.
\end{abstract}


\end{document}